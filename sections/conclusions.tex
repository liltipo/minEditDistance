Los resultados obtenidos a lo largo de este trabajo permiten validar la efectividad de los algoritmos implementados para el cálculo de la distancia de edición. A través de los experimentos realizados, se ha evidenciado que, aunque en teoría la Programación Dinámica ofrece una ventaja computacional sobre la Fuerza Bruta debido a su reducción en el número de operaciones redundantes, en ciertos casos prácticos, la teoría no fue cumplida. Esto puede deberse a las características particulares de las cadenas de prueba utilizadas, así como a las optimizaciones internas en la implementación.\\

\noindent El análisis de los tiempos de ejecución promedio revela que, para cadenas cortas o con patrones simples, ambos enfoques presentan un rendimiento similar. Sin embargo, en casos más complejos, con cadenas más largas o de mayor desorganización, el tiempo de ejecución de la Fuerza Bruta tiende ser un tanto menor al de Programación Dinámica. Esto yendo en contra de la expectativa teórica.\\

\noindent Por otro lado, los resultados muestran que la Programación Dinámica, en algunos casos se comporta mejor en términos de tiempo de ejecución, pero como observamos, no ofrece una mejora sustancial en todos los escenarios. Esto refuerza la idea de que la elección del algoritmo óptimo depende del contexto específico del problema, y que no existe una solución universalmente superior.\\

\noindent En términos generales, el trabajo ha logrado cumplir con el objetivo de evaluar ambos enfoques bajo distintas condiciones, aportando una mejor comprensión de sus fortalezas y limitaciones en la práctica. Los hallazgos obtenidos son coherentes con las expectativas teóricas, pero también ofrecen perspectivas valiosas sobre la importancia de considerar características particulares del problema al elegir un enfoque algorítmico.