\epigraph{\textit{``Indeed, brute force is a perfectly good technique in many cases; the real question is, can we use brute force in such a way that we avoid the worst-case behavior?''}}{--- \citeauthor{taocv3}, \citeyear{taocv3} \cite{taocv3}}

 El algoritmo realiza una búsqueda recursiva exhaustiva, donde en cada paso calcula los resultados de todas las operaciones posibles y selecciona la que tiene el menor costo. Este enfoque es ineficiente, ya que repite los mismos cálculos para las subcadenas muchas veces. La complejidad del algoritmo crece exponencialmente debido a la gran cantidad de posibles combinaciones que se deben considerar. \\

\begin{algorithm}[H]
    \SetKwProg{myproc}{Procedure}{}{}
    \SetKwFunction{EditDistance}{EditDistance}
    \SetKwFunction{Cases}{Cases}
    
    \DontPrintSemicolon
    \footnotesize

    % Definición del algoritmo principal
    \myproc{\EditDistance{S1, S2}}{
        \uIf{LENGTH(S1) es vacía}{
            \Return LENGTH(S2)\;
        }
        \uElseIf{LENGTH(S2) es vacía}{
            \Return LENGTH(S1)\;
        }
        \Return \Cases{S1, S2, 0, 0}\;
    }

    % Definición de la función auxiliar
    \myproc{\Cases{S1, S2, i, j}}{
        \uIf{LENGTH(S1) es igual a i}{
            \Return LENGTH(S2) - j\;
        }
        \uElseIf{LENGTH(S2) es igual a j}{
            \Return LENGTH(S1) - i\;
        }
    
        $costo \leftarrow 0$\;
    
        \uIf{S1[i] == S2[j]}{
            $costo \leftarrow \Cases{S1, S2, i + 1, j + 1}$\;
        }
        \Else{
            $inserción \leftarrow \Cases{S1, S2, i, j + 1}$\;
            $eliminación \leftarrow \Cases{S1, S2, i + 1, j}$\;
            $sustitución \leftarrow \Cases{S1, S2, i + 1, j + 1}$\;
            $transposición \leftarrow \infty$\;
    
            \uIf{i + 1 < ~LENGTH(S1) \textbf{and} j + 1 < ~LENGTH(S2) \textbf{and} S1[i] == S2[j + 1] \textbf{and} S1[i + 1] == S2[j]}{
                $transposición \leftarrow \Cases{S1, S2, i + 2, j + 2}$\;
            }

           $costo \leftarrow \min(inserción, eliminación, sustitución, transposición) + 1$\;
        }
        \Return costo\;
    }
    \caption{Algoritmo con un enfoque de fuerza bruta basado en recursión}
    \label{alg:mi_algoritmo_1}
\end{algorithm}
\newpage
