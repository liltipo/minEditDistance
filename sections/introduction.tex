El análisis y diseño de algoritmos es un campo fundamental en las Ciencias de la Computación, dado que permite resolver problemas complejos de manera eficiente, optimizando el uso de recursos como el tiempo y la memoria. Dentro de este campo, el cálculo de la distancia de edición (o distancia de Levenshtein) es un problema clásico que se utiliza para medir el grado de similitud entre dos secuencias de caracteres. Este problema tiene aplicaciones directas en diversas áreas, como el procesamiento de lenguaje natural y los sistemas de corrección ortográfica, lo que subraya su relevancia \cite{jurafsky2024}. \\

\noindent A lo largo del tiempo, se han propuesto diversos enfoques para resolver el problema de la distancia de edición. El más simple de estos es la Fuerza Bruta, que explora todas las posibles transformaciones entre las dos cadenas, lo que resulta en un algoritmo con una complejidad exponencial \cite{levenshtein1966}. Por otro lado, la Programación Dinámica, introducida en el ámbito de los algoritmos por Bellman \cite{bellman1957}, ofrece una solución más eficiente en términos de tiempo, reduciendo la complejidad a O(n*m) mediante el uso de una tabla que almacena subproblemas previamente resueltos. A pesar de que la Programación Dinámica es teóricamente más eficiente, en ciertos escenarios prácticos esta diferencia de rendimiento puede no ser tan evidente debido a las características específicas de los datos o implementaciones. \\

\noindent El propósito de este informe es evaluar ambos enfoques —Fuerza Bruta y Programación Dinámica— para el cálculo de la distancia de edición, comparando su rendimiento en términos de tiempo de ejecución y cantidad de operaciones necesarias. Además, se busca analizar si las implementaciones de estos algoritmos cumplen con las expectativas teóricas en cuanto a eficiencia y complejidad, como se detalla en la obra de Cormen et al. \cite{cormen2009}. \\

\noindent Este informe aborda una pregunta clave: ¿hasta qué punto las optimizaciones teóricas de la Programación Dinámica se traducen en mejoras prácticas frente a la Fuerza Bruta? El análisis de estos algoritmos no solo permitirá contrastar sus desempeños en distintos escenarios, sino también comprender las implicaciones de seleccionar un enfoque algorítmico sobre otro, dependiendo de las características del problema. \\

\noindent A lo largo del desarrollo del informe, se expondrán los detalles de la implementación de ambos algoritmos, los experimentos realizados para evaluar su desempeño y los resultados obtenidos. Se espera que estos análisis proporcionen una comprensión más profunda de los factores que influyen en la eficiencia de los algoritmos de distancia de edición y cómo estos se comportan en la práctica, más allá de sus complejidades teóricas.